\documentclass[a4paper,10pt,dvips]{article}
\usepackage{hyperref}
\usepackage[cp1252]{inputenc}
\usepackage[T1]{fontenc}
\usepackage{lmodern}
\usepackage{geometry}
\geometry{hmargin=2cm,vmargin=2cm}
\usepackage{graphicx}
\usepackage{pstricks,pst-plot,pst-tree}
%\usepackage{pst-eucl}
\usepackage{framed}
\usepackage{amsmath,bm}
\usepackage{amssymb}
\usepackage{fancyhdr}
\usepackage{fancybox}
\usepackage{multicol}
\usepackage{xcolor}
\usepackage{epsfig}
\usepackage{pifont}
\usepackage[framed]{ntheorem}
\usepackage[frenchb]{babel}
\usepackage{tabularx}


\lhead[]{\small classe ISI}
\chead[]{\textsc{G�om�trie dans l'espace}}
\rhead[]{\small date}
\lfoot{\small \url{http://mathematiques.daval.free.fr}}
\rfoot{\small -\thepage-}
\cfoot{}

\def\R{{\mathbb R}}
%\def\Q{{\mathbb Q}}
%\def\Z{{\mathbb Z}}
%\def\D{{\mathbb D}}
%\def\N{{\mathbb N}}
%\def\C{{\mathbb C}}

\pagestyle{fancy}


\def\theoremframecommand{\psshadowbox*[framearc=0.5]}
\newshadedtheorem{Partie}{Partie}
\theoremstyle{break}
\theorembodyfont{\upshape}
%\newframedtheorem{Prop}{Propri�t�}
%\newtheorem{Rem}{Remarque}
\theorembodyfont{\rmfamily}
%\theorempostwork{\hrule}
\newtheorem{Exo}{\ding{50}~~Exercice}

%\theorembodyfont{\small \sffamily}
%\newtheorem{Ex}{Exemple}
%\theorembodyfont{\slshape}
%\newframedtheorem{Def}{D�finition}
%\renewcommand\thesection{\Roman{section}}
\renewcommand\FrenchLabelItem{\textbullet}
\newcommand{\V}{\overrightarrow}
\newcommand{\Rep}{(O;\V{i};\V{j})}
%\newcommand{\Int}[2]{[\, #1 ; #2 \,]}
%\newcommand{\Reel}{\mathfrak{Re}}
%\newcommand{\Ima}{\mathfrak{Im}}
%\newcommand{\Abs}[1]{\left \lvert#1\right \rvert}
\newcommand{\Coor}[2]{\begin{pmatrix} #1\\#2 \end{pmatrix}}
\newcommand{\Syst}[2]{\left\{\begin{array}{ccccc} #1\\ #2 \end{array}\right.}
\newcolumntype{C}[1]{>{\centering\arraybackslash}p{#1cm}}
%\renewcommand{\arraystretch}{1.2}
%\renewcommand{\baselinestretch}{1.2}

%\setlength{\parskip}{3ex plus 1ex minus 1ex}
%\setlength{\parindent}{0cm}

\begin{document}


\textbf{\begin{center}
\shadowbox{\Huge{Devoir surveill� n�12}}
\end{center}}

\smallskip

%%%%%%%%%%%%%%%%%%%%%%%%% EXO 1 %%%%%%%%%%%%%%%%%%%%%%%%%%

\begin{Exo}
Donner le nombre de faces de chacun des solides suivants :\\

\scalebox{0.5}
{
\begin{pspicture*}(0,-2)(6,6)
   \pspolygon[linecolor=black,fillcolor=lightgray,fillstyle=solid](0,0)(0,2)(2,4)(4,4)(4,0)
   \pspolygon[linecolor=black,fillcolor=lightgray,fillstyle=solid](4,4)(6,6)(6,2)(4,0)
   \pspolygon[linecolor=black,fillcolor=lightgray,fillstyle=solid](6,6)(2,6)(1,5)(2,4)(4,4)
   \pspolygon[linecolor=black,fillcolor=lightgray,fillstyle=solid](1,5)(0,2)(2,4)
\end{pspicture*}
}
\hspace{2cm}
\scalebox{0.5}
{
\begin{pspicture*}(0,-1)(7.54,9.5)
   \pspolygon[linecolor=black,fillcolor=lightgray,fillstyle=solid](3.02,0.98)(3.02,1.98)(4.02,2.98)(4.02,7.98)(5.02,8.98)(5.02,6.98)(7.02,8.98)(7.02,3.98)(3.02,0)
   \pspolygon[linecolor=black,fillcolor=lightgray,fillstyle=solid](3.02,0)(2.02,0)(2.02,1.98)(3.02,2.98)(3.02,5.98)(0.02,5.98)(0.02,6.98)(3.02,6.98)(3.02,7.98)(4.02,7.98)(4.02,2.98)(3.02,1.98)
   \pspolygon[linecolor=black,fillcolor=lightgray,fillstyle=solid](0.02,6.98)(1.02,7.98)(3.02,7.98)(3.02,6.98)
   \pspolygon[linecolor=black,fillcolor=lightgray,fillstyle=solid](3.02,7.98)(4.02,8.98)(5.02,8.98)(4.02,7.98)
   \pspolygon[linecolor=black,fillcolor=lightgray,fillstyle=solid](7.02,8.98)(6.02,8.98)(5.02,7.96)(5.02,6.98)
   \psline(2,2)(3,2)\psline(3,3)(4,3)
\end{pspicture*}
}
\hspace{2cm}
\scalebox{0.5}
{
\begin{pspicture*}(0,-2)(9,7)
   \pspolygon[linecolor=black,fillcolor=lightgray,fillstyle=solid](3,0)(8,0)(8,6)(7,6)(7,3)(5,3)(5,4)(3,4)(2,3)(2,2)(3,3)
   \pspolygon[linecolor=black,fillcolor=lightgray,fillstyle=solid](-0,2)(2,2)(2,3)(-0,3)
   \pspolygon[linecolor=black,fillcolor=lightgray,fillstyle=solid](-0,3)(2,5)(6,5)(5,4)(3,4)(2,3)
   \pspolygon[linecolor=black,fillcolor=lightgray,fillstyle=solid](6,5)(6,4)(7,4)(7,3)(5,3)(5,4)
   \pspolygon[linecolor=black,fillcolor=lightgray,fillstyle=solid](8,0)(9,1)(9,7)(8,7)(7,6)(8,6)
   \psline(5,3)(6,4)
   \psline(8,6)(9,7)
\end{pspicture*}}


\end{Exo}


%%%%%%%%%%%%%%%%%%%%%%%%% EXO 2 %%%%%%%%%%%%%%%%%%%%%%%%%%
\begin{Exo}
ABCDEFGH est un pav�, I est le mileiu de [EF] et J le milieu de [HG]. \\
\begin{center}
  \begin{pspicture}(-1,0)(5,3)
      \psset{xunit=1.8}
      \pspolygon(0,0)(2,0)(3,1)(3,3)(1,3)(0,2)
      \psline(0,2)(2,2)(2,0)
      \psline(2,2)(3,3)
      \psline[linestyle=dashed](0,0)(1,1)(3,1)
      \psline[linestyle=dashed](1,1)(1,3)
      \psdots(1,2)(2,3)
      \rput(-0.2,-0.2){A}
      \rput(2,-0.2){B}
      \rput(1.1,2.2){I}
      \rput(2.1,3.2){J}
      \rput(2.2,2){F}
      \rput(-0.2,2){E}
      \rput(0.8,1){D}
      \rput(3.2,1){C}
      \rput(3.1,3.2){G}
      \rput(1,3.2){H}
   \end{pspicture}
\end{center}

\begin{enumerate}
   \item Donner respectivement :
   \begin{enumerate}
      \item Une droite parall�le � la droite (IJ), non coplanaire au plan (EHF) et s�cante � la droite (GB).
      \item Un plan parall�le au plan (IJG) et s�cant au plan (EAD).
      \item Une droite parall�le au plan (ABC), s�cante au plan (FGC) et confondue dans le plan (HGF).
   \end{enumerate}
   \item �tudier la position relative des droites suivantes :
   \begin{enumerate}
      \item La droite (BH) et la droite (BC).
      \item La droite(EG) et la droite(BC).
      \item La droite(EG) et la droite(AC).
   \end{enumerate}
   \item Quel est, dans chacun des cas suivants, l'intersection des deux plans :
   \begin{enumerate}
      \item Le plan (EIA) et le plan (FIC).
      \item Le plan (EHI) et le plan (FJG).
      \item Le plan (DAB) et le plan (FJG).
   \end{enumerate}
\end{enumerate}

\end{Exo}

\smallskip

\begin{Exo}
ABCDEFGH est un cube dont l'ar�te mesure 2 cm. P et Q sont les centres respectifs des faces EFGH et BCGF. \\
\begin{minipage}{12cm}
   \begin{enumerate}
      \item Tracer en vraie grandeur le patron du cube (avec les points P et Q).
      \item Calculer EP.
      \item En quoi le triangle AEP est-il rectangle ? Justifier.
      \item En d�duire que AP $=\sqrt{6}$ cm.
      \item En utilisant le triangle BEG, calculer PQ.
      \item Quel nom peut-on donner au solide GEBF ? Calculer alors son volume.
   \end{enumerate}
\end{minipage}
\begin{minipage}{5cm}
   \begin{pspicture}(-1,0)(3,4)
      \psset{unit=1.2}
      \pspolygon(0,0)(2,0)(3,1)(3,3)(1,3)(0,2)
      \psline(0,2)(2,2)(2,0)
      \psline(2,2)(3,3)
      \psline[linestyle=dashed](0,0)(1,1)(3,1)
      \psline[linestyle=dashed](1,1)(1,3)
      \rput(-0.2,-0.2){A}
      \rput(2,-0.2){B}
      \rput(2.2,2){F}
      \rput(-0.2,2){E}
      \rput(0.8,1){D}
      \rput(3.2,1){C}
      \rput(3,3.2){G}
      \rput(1,3.2){H}
      \psline[linestyle=dotted,linewidth=0.05](2,0)(3,3)(0,2)
      \psline[linestyle=dotted,linewidth=0.05](1,3)(2,2)(3,1)
      \rput(1.5,2.7){P}
      \rput(2.7,1.5){Q}
   \end{pspicture}
\end{minipage}
\end{Exo}

\end{document}